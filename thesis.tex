% !TEX encoding = UTF-8 Unicode
% !TEX TS-program = xelatex

\documentclass[a4paper, 12pt]{book}

\usepackage{FDUThesis}
\usepackage{gbt7714} %在正文中 \cite 文献
\author{Yizhan Miao $<$\href{mailto:yzmiao@pm.me}{yzmiao@pm.me}$>$}

\begin{document}
\setlength{\baselineskip}{20pt}  % 20磅行间距

% \includepdf{covers/cover-1.pdf}
% \thispagestyle{empty}
%
% \includepdf{covers/cover-2.pdf}
% \thispagestyle{empty}
%
% \includepdf{covers/cover-3.pdf}
% \thispagestyle{empty}
%
% \includepdf{covers/cover-4.pdf}
% \thispagestyle{empty}

\includepdf{covers/book-cover.pdf}
\thispagestyle{empty}
% use \thispagestyle{} fancy, plain, empty to redefine Per/Page Header

% ---------------- Front Matter ----------------
\frontmatter

\phantomsection
\addcontentsline{toc}{chapter}{\contentsname}
\tableofcontents

\frontchapter{中文摘要}
时间作为一个十分重要的维度, 我们的神经系统对不同跨度时间间隔的编码有不同的机制;
其中, 最为复杂的就是对秒和数秒层面的计时。我们通过对植入有电极的癫痫患者进行
与小鼠实验相同的视觉刺激和相近的行为学测试, 获得人在进行规律性时间间隔预测行为
下的行为和脑电信号。通过对行为学的分析, 结果提示人和小鼠的预测行为需要一定的学习过程,
而人的学习可以在几个刺激内完成。通过对脑电信号的能量谱分析, 我们发现视觉信号主要
引起\(\gamma\)波段的振荡, 而相位分析提示被试能否准确预测时间间隔会引起相位锁定
的不同。最后, 我们基于现有的实验结果, 提出了可能的理论猜想:
时间预测信号在局部场电位层面, 可能通过改变皮层放电潜在的相位变化而实现;
同时, 我们也对小鼠和人的脑电信号做了简单的对比。

\bigskip
\noindent \textbf{关键词: \hspace{\Han}}
时间预测;\;
\(\gamma\)振荡;\;
sEEG;\;
立体定向脑电图;\;
相位分析;\;



\bigskip
\noindent \textbf{中图分类号: \hspace{\Han}Q427}

% ----------------
\frontchapter{英文摘要}
% Time as an important dimension of reality, is encoded by nervous system,
% just like other perceptions. For different ranges of time intervals,
% nervous system has different mechanisms. Timing of seconds and minutes
% is one of the most complicated timing, and the mechanism is still unclear.
% Here, we used patients who had epilepsy with sEEG electrodes implantation
% as our subjects. We used the identical visual stimuli and similar behavior
% paradigms used in mice experiments for the subjects. The behavioral resutls
% implied that a learning process might be involed in our paradigms of both humen and mice.
% However, human subjects could learn the time interval within only few trials.
% Then, we analyzed the power spectra of sEEG recording, and we found that
% the visual stimuli caused \(\gamma\) oscillation most. And further phase analysis
% showed difference of phase locking between correct prediction of time interval
% and false prediction. Finally, based on the current results, we purposed
% a possible hypothesis of mechanism of our results: in local field potential,
% the prediction of time interval  might be encoded in the phase of recording signal.
Time as an important dimension of reality is encoded by the nervous system,
just like other perceptions. For different ranges of time intervals,
the nervous system has different mechanisms.
The timing of seconds and minutes is one of the most complicated timing,
and the mechanism is still unclear.
Here, we used patients who had epilepsy with sEEG electrodes implantation as our subjects.
We used identical visual stimuli and similar behavior paradigms used in mice experiments for the subjects.
The behavioral results implied that a learning process might be involed in our paradigms of both humans and mice.
However, human subjects could learn the time interval within only a few trials.
Then, we analyzed the power spectra of sEEG recording,
and we found that the visual stimuli caused γ oscillation most.
And further phase analysis showed the difference of phase locking
between the correct prediction of the time interval and false prediction.
Finally, based on the current results, we purposed a possible hypothesis of the mechanism of our results, that,
in local field potential, the prediction of time interval might be encoded in the phase of a recording signal.

\bigskip
\noindent \textbf{Key Words:\hspace{\Han}}
time prediction;\;
\(\gamma\) oscillation;\;
sEEG;\;
stereoelectroencephalograhpy;\;
phase analysis

\bigskip
\noindent \textbf{CLC Number:\hspace{\Han}Q427}


%\listoffigures

% ---------------- Main Matter ----------------
\mainmatter

\chapter{前言}

中枢神经系统对时间的编码和感知与生物的行为和感觉密切相关。
而对每一个个体的生存而言, 预判未来可能发生的事件更是至关重要。
过去的研究主要集中于模式动物对时间的感知与预测上;
由于技术手段和伦理等的限制, 对人脑的研究相对较少。
%另一方面, 过去的研究大多将感觉与运动的计时信号分别对待研究;
%而最近几年的科学研究对感觉和运动的信息整合愈加重视。

对于中枢神经系统对时间的感知, 大致可分为两个大的派系:
专用模型(dedicated model)和内在模型(intrinsic model);
而越来越的实验证据与内在模型的描述相符合\cite{paton2018neural}。
对于内在模型, 最显著的一个特点就是时间的感知和编码是由神经元群体(population)来完成,
而群体内每一个单元的动态状态在时间尺度上共同组成了神经活动轨迹(neural trajectory)
\cite{buonomano2009state, remington2018dynamical}。

最初, 人们认为初级视皮层只参与了视觉感觉信息处理的最初阶段。
但随着研究的深入, 有越来越多的证据表明初级视皮层可能也参与了其他高级的脑功能。
Salvioni等人通过经颅磁刺激(TMS)技术探究了正常人初级视皮层和纹外皮层对计时相关活动的作用;
发现抑制了视皮层后, 人在进行与计时相关的视觉任务时,
正确率会显著下降; 而与计时无关的视觉任务则不受影响\cite{salvioni2013visual}。
在非人灵长类动物的实验中, Lima 等人发现初级视皮层的局部场电位(Local Field Potential, LFP)
会与动物计时行为出现相关的变化,
包括\(\alpha\)波段振荡的减弱和\(\gamma\)波段振荡的增强\cite{lima2011gamma}。
而在啮齿类模式生物中, Shuler等人发现在初级视皮层中存在与奖励时间间隔相关的
神经活动以及场电位变化\cite{chubykin2013cholinergic, shuler2006reward, zold2015theta};
之后, 他们又通过药物阻断、膜片钳、光遗传等手段进一步验证了乙酰胆碱能神经元
输入对初级视皮层的增强学习过程有决定性的作用\cite{chubykin2013cholinergic, liu2015selective, namboodiri2015visually}。

此课题我们将研究的重点关注在相同视觉刺激下人脑和小鼠脑
局部场电位(LFP)的比较, 以及人脑场电位在时间预测性行为下的变化。
本课题将依托导师与华山医院的合作, 收集sEEG术后癫痫患者的脑电数据,
并平行地与过去相同范式刺激下的小鼠电生理数据做比较。


\chapter{材料与方法}

\section{实验材料}

\subsection{实验对象}
我们总共采集了6位患者的脑电数据, 均为男性(表\ref{tab:patient_info})。
患者由合作方复旦大学华山医院提供, 每位患者均有签订书面的知情同意书,
对实验的细节与科研用途表示支持与同意。本课题也符合华山医院伦理委员会的相关规定与同意。
每位患者均为癫痫病患者, 因手术治疗准备而在其颅内植入了数根医用不锈钢电极。
在实验期间, 患者均有服用抗癫痫药物, 且无癫痫发作。

由于患者病情发展, 以及癫痫病灶引起的并发症不一, 实际有效的病人为3人,
有效的通道数目共计268个。

\begin{table}[h]
    \centering
    \caption{被试基本信息}
    \label{tab:patient_info}
    \begin{tabular}{ccccccc}
        \hline\noalign{\smallskip}
        被试 & 性别 & 年龄 & 癫痫病灶 & 记录电极通道数目 & 实验时间(天) & 能否有效完成实验\\
        \hline\noalign{\smallskip}
        \#1 & 男 & 24 & 枕叶 & 112 & 7 & +\\ % zhouchen
        \#2 & 男 & 14 & 顶叶 & 111 & 2 & -\\ % shuyunfan
        \#3 & 男 & 26 & 颞叶 &  73 & 7 & +\\ % fansulong
        \#4 & 男 & 23 & 颞叶 & 138 & 7 & -\\ % wuzhenwei
        \#5 & 男 & 28 & 颞叶 &  91 & 7 & -\\ % zhangchen
        \#6 & 男 & 24 & 颞叶 &  83 & 7 & +\\ % wangjinan

        \hline\noalign{\smallskip}

    \end{tabular}
\end{table}

行为实验的对照组邀请了同实验室的3位年龄与患者相仿的男性同学参与;
对照组成员均自愿同意参与本课题, 且仅做了行为学部分的实验。
为了排除对照组被试同学对本课题了解的先入影响,
每位实验对象对间隔为13~秒的视觉刺激进行了实验, 连续测试了3~次, 仅测试1~天。

\subsection{实验动物}
动物实验部分的数据由张嘉漪课题组博士生于庆鹏完成并提供。
本实验中使用的所有动物都符合复旦大学上海医学院动物管理机构及使用委员会的相关规定,
并参照美国国立卫生研究院的实验动物管理及使用标准。
本课题中使用的实验动物均为8-16周雄性C57/BL6J小鼠, 购买自上海斯莱克实验动物有限责任公司。
用于后续数据分析的共计两只小鼠。

\subsection{实验仪器}

\subsubsection{脑电采集系统}
患者脑电数据的采集依赖于华山医院的数据采集系统,
Neurofax EEG-1200C采集系统(NIHON KOHDEN Corporation, Japan)。
脑电原始数据采样频率选用了2~kHz, 并经过了采集系统自带的0.5Hz到600Hz的带通滤波(bandpass filter),
以及50Hz的陷波滤波(notch filter)。

\subsubsection{小鼠电生理采集系统}
本课题中小鼠部分电生理数据采集使用了Spike2记录系统(Cambridge Electronic Design Limited, UK)。
主要采集了小鼠的局部场电位信号(Local Field Potential, LFP)。
采样频率为10~kHz, 为与后续患者数据分析相匹配,
我们人为地将采集到的数据降频至2~kHz。

\subsubsection{视觉刺激系统}
视觉刺激主要依靠自制程序实时生成。
对患者的实验中, 视觉刺激主要由Macbook Pro(13 inches, 2013-late, Apple Inc.)呈现。
屏幕位于患者的正前方, 距离患者眼睛约50~cm。
小鼠实验中, 视觉刺激主要由小显示器(7英寸)呈现, 屏幕位于小鼠的一侧眼前,
小鼠瞳孔与屏幕中心点的连线与屏幕平面垂直, 距离小鼠眼睛约10~cm。

%实验对象
%实验动物
%实验仪器

\section{实验主要方法}
% 实验方法
%% [x] 视觉刺激与行为范式
%% [x] 行为学结果分析
%% [ ] 电极定位
%% [x] 能量谱分析
%% [x] 能量曲线
%% [ ] 支持向量机
%% [ ] 相位分析
%% [x] 代码的获取

\subsection{视觉刺激与行为范式}

我们利用自制的Python程序,依托PsychoPy软件\cite{psychopy}实时生成视觉刺激。
刺激主体为移动光栅(drifting grating),每次视觉刺激持续1秒钟;
光栅为方波光栅,对比度为100\%,时间频率为2Hz,空间频率为0.05周期每度(cycles per degree)。

我们共设计了3种行为范式,分别为“轻拍手”,“默想”,和“空想”。%TODO 图例连接
“轻拍手“范式要求患者主动预判视觉刺激出现的时刻,并尽可能地在视觉出现前轻按电脑的空格键以记录患者的预判时间。
“默想”范式要求患者主动预判时间刺激的时刻,但无需做出行为而只需要在脑海中默念。
“空想”范式要求患者放空思想,以尽量减少主观思考和注意力的影响。

每种范式又分别有两种时间间隔,分别为5秒和10秒;
对于5秒间隔,视觉刺激持续1秒,50\%灰背景持续4秒;
对于10秒间隔,视觉刺激持续1秒,50\%灰背景持续9秒。
每种范式和每个时间间隔刺激均为20次,刺激结束前后均有20秒的50\%灰背景。

每位患者每天只进行一次实验,每次的顺序均为“轻拍手”,“默想”,和“空想”范式,
每个范式均为先10秒间隔后5秒间隔,两次间隔间有约30到60秒的休息时间。

\subsection{行为学数据分析}

\subsubsection{韦伯定律}
为了量化实验对象的行为学,同时消除不同时间间隔间的差异,
我们计算了行为学的韦伯系数(Weber's Coefficient)\cite{gibbon1977scalar, hardy2018encoding}。具体公式如下:
\begin{equation}
    W = \frac{\sqrt{\frac{1}{N} \sum_{i=1}^N (x_i - \mu) ^ 2}}{\mu},\ \mathrm{where}\ \mu = \frac{1}{N} \sum_{i=1}^N x_i
\end{equation}

其中,N表示所观察的测试总次数,x为每次轻拍手与前一次光栅刺激开始的相对时间,反应了患者需要主动计时的时间长度,
即$ x = t_{tapping} - t_{previous\ grating\ onset} $。

\subsubsection{行为分布图}
为了更细致的体现患者行为中可能存在的学习过程,我们对行为的分布做了进一步的分析。
我们计算了每次轻拍手的时刻与相应光栅出现时刻的差值作为行为分布图分析用的数据集,即$\Delta t = t_{tapping} - t_{grating\ onset}$。
我们将前后相邻的两次行为作为一组数据,前一次轻拍手的时刻作为横轴,后一次平拍手的时刻作为竖轴。
对于每一天的一个时间间隔而言,共有19对这样的数据。我们将19对数据如上述分成了5份(第一份为3对,其余为4对)。
为了减少偶发事件对分析的影响,我们将7天相对应的每份合并在一起;即第一份共有21对,其余4份各有28对。

不同象限可以反应当时患者行为的不同状态。第一象限表示测试对象出现连续两次延后于光栅的轻拍手;
第三象限表示被试出现连续两次提前于光栅的轻拍手;
而第二、四象限则提示被试两次轻拍手各有一个在光栅前和后。

\subsection{脑电数据处理}

\subsubsection{电极通道定位}

每个电极通道在脑中的定位主要依靠术前的MRI和术后的CT成像(影像数据由华山医院提供)。
我们首先利用FSL程序\cite{fsl}将影像配准到MNI标准空间,并手动标记每个电极的位置。
我们再利用MNI空间至Talairach空间转换公式\cite{bioelectromagnetism} %TODO bibtex链接有问题
获得电极通道相对应的Talairach座标。
之后再利用Talairach Daemon软件\cite{talairach_daemon}获得每个通道相对应的脑区位置。
为进一步方便分析,我们将各个脑区再根据各自的生理功能大致做了合并(见正文图表)。%TODO 图例链接

\subsubsection{能量谱分析}

我们使用自制的Python3程序,利用scipy\cite{scipy}与numpy\cite{numpy,oliphant2007python}对原始SEEG数据进行后续分析处理。
我们首先将原始数据以时间刺激开始的时刻作为原点,前后各3秒(5秒间隔)或6秒(10秒间隔)进行切割。
之后,对各个片段做以复数形式Morlet小波为基础的小波变换。
\begin{equation}
    %def morlet(F, fs):
    %   """Morlet wavelet"""
    %   wtime = np.linspace(-1, 1, 2*fs)
    %   s = 6 / (2 * np.pi * F)
    %   wavelet = np.exp(2*1j*np.pi*wtime*F) * np.exp(-wtime**2/(2*s**2))
    %   return wavelet
    M(f, t) = e ^ {2 \pi f t i} \cdot e ^ {-\frac{t^2}{2 * (6 / 2 \pi f)^2}}
\end{equation}
其中f为小波的震荡频率,t为相对应的时刻点;t的取值范围为-1到1秒,间隔为0.0005秒。
震荡频率f的变换范围为对数分布的1到150Hz,共计40个频率点。
将变换后的绝对值的对数形式做Z score标准化,选取刺激前的3秒到刺激前的1秒作为基线计算标准化所需要的均值与方差;
并将当天标准化后的数值平均后作为能量谱进行做图。

\subsubsection{能量曲线分析}
为了方便后续的量化分析,并进一步了解特定波段范围脑电,我们对脑电的能量曲线做了分析。
我们首先对原始数据依照相应的频段范围进行了带通滤波,主要为$\gamma$波段(30-80 Hz)。
对于带通滤波,我们选用了有限冲激响应滤波(finite impulse response filter)。
对于滤波后的数据,再经过了Hilbert转换,将常量数据转换为以下形式:
\begin{equation}
    S(t) = A(t) \cdot e^{i \theta(t)}
\end{equation}
其中S为变换后的数据,$A$和$\theta$为欧拉形式的两个参数,t为相应的时刻点。
将变换后的绝对值的对数形式做Z score标准化,
选取刺激前的3秒到刺激前的1秒作为基线计算标准化所需要的均值与方差。
标准化后的曲线即为相对应波段的能量曲线。


% \subsubsection{回现(entrain)与光反应的自动检测}
%
% 原始数据首先通过FIR带通滤波对每个病人和特定脑区特征的频段进行滤波。
% 滤波后的信号进一步通过Hilbert转化并进行了如上述的切割。
% 将变换后的绝对值的对数形式做Z score标准化,并将当天标准化后的值平均后作为其特征波段的能量值(power)。
%
% 只有在有效窗口(刺激开始的时刻前后1秒或2秒)内能量值大于1.96,
% 同时在非有效窗口的能量值小于1.96的通道或者试验才能被认为有光反应或出现了回现(entrain)。
%
% \subsubsection{回现(entrain)的随机水平(chance level)}
%
% 我们将同一范式,两次间隔试验的间隔作为基线(时长约为30到60秒)。
% 而后,我们随机生成1000个刺激起始时刻,重复上述的自动检测。
% 被认定为有出现了回现的次数占总测试次数的比例即为回现的随机水平。

% \subsubsection{支持向量机(support vector machine, SVM)分析}
% 为了进一步量化神经元群体和不同脑区间对时间间隔的编码,
% 我们利用径向基核函数多分类支持向量机进行评估(multiclass radial-basis function kernel support vector machine )。
% 我们将时间间隔分割为20份(对于5秒间隔,每份为250毫秒,对于10秒间隔则为500毫秒)。
% SVM的实现依赖于LIBSVM库(DOI: 10.1145/1961189.1961199)。
% 多分类采用了one-against-one的方法,SVM的输出为一个20维的向量,代表了SVM对于输入数据与对应的20个时刻的相似程度。
% SVM的训练与测试采用了留一交叉验证(Leave-One-Out Cross Validation)的方法,对最后的SVM结果利用Pearson相关系数来做量化。
%
% 我们严格把控和限制了支持向量机的数据集。我们仅选择了在视皮层,扣带回的相关脑区,并将其安装有无光反应做了分类。
% 同时,为了增加对照并评估过拟合的可能,我们还对数据采取了试验内随机化和试验间随机化。
% 试验内随机化将同一次试验内的不同时间做随机化;试验间随机化则将相同时间点但不同试验的数据做随机化。
% 随机化后的SVM结果也用Pearson相关系数来做量化。

\subsubsection{相位分析}
相位也是脑电信号的重要组成部分之一。为了量化患者脑电信号的相位,
我们选择了计算脑电信号的ITPC(Inter Trial Phase Cluster)值\cite{gu2010phase}。
计算公式为:
\begin{equation}
    % ITPC = \left\| \frac{\sum_{k=1}^N e^{i \theta_k}}{N} \right\|
    ITPC(t) = \frac{1}{N} \left\| \sum_{k=1}^N e^{i \theta_k(t)} \right\|
\end{equation}
其中N为所观察的测试总次数,$\theta$为Hilbert转换后的欧拉形式参数,t为相对应的时刻点。

%TODO 额外的对ITPC的解释,包括其物理意义

\subsubsection{分析用代码}
视觉刺激用代码如有需要,请与本人或指导老师联系,如理由正当合理将以邮件形式分享。
后续数据分析用Python代码可在\href{https://github.com/ZhangJiayiLab/EEGAnalysis}{GitHub}获得。

%行为范式
%视觉刺激
%数据分析

% 实验材料
%% 实验对象
%% 实验小鼠
%% 实验器材

% 实验方法
%% 视觉刺激与行为范式
%% 行为学结果分析
%% 电极定位
%% 能量谱分析
%% 能量曲线的获得
%% 支持向量机
%% 相位分析
%% 代码的获取

\chapter{实验结果}
% outline
% 行为学结果
% - [x] 小鼠行为学结果
%   - [x] 行为范式说明
% - [ ] 患者行为学结果
%   - [x] 行为范式说明
%   - [x] 韦伯定律
%   - [x] 行为分布图
%   - [ ] 行为学提示所需的任务可能与增强学习相关
% 电生理结果
% - [ ] 能量谱提示主要的波段
% - [ ] 能量曲线的变化
% - [ ] 相位分析显示学习过程
% 机器学习与可能的理论机制
% - [ ] 支持向量机对刺激间隔的解码
% - [ ] 对不同通道相位变化的降维分析

% 行为学结果
% - [ ] 小鼠行为学结果
%   - [ ] 行为范式说明
% - [ ] 患者行为学结果
%   - [ ] 行为范式说明
%   - [ ] 韦伯定律
%   - [ ] 行为分布图
%   - [ ] 行为学提示所需的任务可能与增强学习相关

\section{小鼠的时间间隔预测行为}
为了探究视觉信息对小鼠的时间感知的影响,尤其是对视觉刺激所编码的时间的感知,
我们设计了连续周期性视觉刺激时间预测行为实验。通过耦连视觉刺激与小鼠舔水行为,
我们可以检测小鼠的舔水来推断小鼠内在的计时。

\subsection{规律刺激下的小鼠时间预测行为}
% 行为学范式示意图
% 一天舔水的示例
% 第一次舔水的分布曲线
% weber定律
我们选用了经过限水处理的野生型C57小鼠作为实验对象,每天包含5次连续的训练,
每次训练包含20个规律性的刺激。每次视觉刺激持续1秒,刺激间间隔为9秒;
每次刺激开始时小鼠均可以获得一定的水作为正向的反馈(图\ref{fig:mouse_behavior})。

为了评判小鼠在一天中的行为表现,我们将一天中小鼠舔水的行为时刻做了记录;
并以每次视觉刺激的开始作为原点,将前8秒和后2秒作为事件窗口对舔水行为做了分组。
同时,我们将不同组的行为以第一次舔水的时刻点做了排序(图\ref{fig:mouse_behavior})。
为了进一步比较不同天之间小鼠的行为表现,我们将第一次舔水的时刻点单独拿出来,
并对在视觉刺激之前发生的行为做了累计分布曲线(图\ref{fig:mouse_behavior})。
我们可以看到随着训练天数的增加,小鼠的舔水时刻向刺激开始的时间靠近。
前三天和后三天舔水的平均分布之间也存在统计学上的显著差异
(Kolmogorov–Smirnov检验,p值为$2.3 \times 10 ^ {-41}$), 提示了
小鼠在训练过程中存在着学习过程, 对刺激间时间间隔的感知准确度有所提高。

另一方面,我们也通过计算了小鼠行为学的韦伯系数(weber's fraction)
来检验小鼠对时间间隔的感知精确度(图\ref{fig:mouse_behavior})。
由于时间所限,我们未能对大量的老鼠
来做行为学;但我们依然能够看到随着训练时间的延长,小鼠行为学的韦伯系数
呈下降趋势。

综合小鼠的行为学结果来看,我们可以推断在连续周期性视觉刺激时间预测行为实验中,
小鼠对刺激间时间间隔的感知的精确度和准确度均有一定的提升,提示小鼠可能存在
学习过程;而小鼠大脑中神经元之间是否有发生动态的变化,还需要进一步的电生理实验
来验证。

\begin{figure}[h]
    \begin{center}

    \includegraphics[width=0.45\textwidth]{src/figures/mouse_behavior_schema.pdf}
    ~
    \includegraphics[width=0.45\textwidth]{src/figures/mouse_behavior_example.pdf}

    \includegraphics[width=0.45\textwidth]{src/figures/mouse_behavior_curve.pdf}
    ~
    \includegraphics[width=0.45\textwidth]{src/figures/mouse_behavior_weber.pdf}
    \end{center}

    \caption{\textbf{规律刺激下的小鼠时间预测行为}\\
    一些额外的图注说明}
    \label{fig:mouse_behavior}
\end{figure}

%%%%%%%%%%%%%%%%%%%%%%%%%%%%%%%%%%%%%%%%%%%%%%%%%%%%%%%%%%%%%%%%
\section{患者的时间间隔预测行为}
为了让对人的实验结果与小鼠的实验结果可以

\subsection{心理测试范式}

\subsection{韦伯定律}

\subsection{轻拍手行为分布}

\subsection{行为结果所体现的可能的理论机制}

\begin{figure}[h]
    \centering
    \includegraphics[width=0.9\textwidth]{src/figures/human_behavior_distribution_control.png}

    \includegraphics[width=0.9\textwidth]{src/figures/human_behavior_distribution_patient_6.png}

    \caption{规律刺激下的人的时间预测行为}
    \label{fig:human_behavior}
\end{figure}







% 电生理结果
% - [ ] 能量谱提示主要的波段
% - [ ] 能量曲线的变化
% - [ ] 相位分析显示学习过程

\section{电生理记录与脑电频谱变化}
由于临床需要,医用不锈钢电极被放置在每位患者主要癫痫病灶区域,
用以监测和定位癫痫的起始位置。通过与华山医院的合作,以及患者
的同意,我们得以在患者进行行为学测试的同时,记录下相应电极的
电信号。癫痫多发于颞叶,而电极的主要记录位置因而集中在颞叶;
但其中有一例特别的病例,患者\#1的癫痫病灶集中在枕叶,因而得以
记录到来自视觉皮层的脑电信号(图\ref{fig:ephys_example}~a)。

\subsection{能量谱提示主要的波段}
我们首先对原始的数字信号进行了小波变化处理,将每次光栅刺激的脑电信号
依据光栅开始的时间对齐,做出脑电信号频谱图(图\ref{fig:ephys_example}~c)。
通过频谱图可见,患者脑电的主要频谱集中在50至100赫兹,即\(\gamma\)波段(\(30 \sim 80\ \text{Hz}\))。

为了进一步量化脑电信号能量的变化,我们依据频谱获得的频率范围以及\(\gamma\)波段范围,选取了
相对应的50至80赫兹作为后续的探究范围。我们计算得到经过带通滤波的脑电信号的能量变化,
并经过z-score归一化得到每个通道的能量曲线(图\ref{fig:ehpys_example}~d)。
与能量频谱相同,能量曲线在光栅开始后也出现了明显的增强,并横跨了整个刺激期间。

我们对小鼠的局部场电位信号也做了相同的分析处理(图\ref{fig:ephys_example}~f, g),
可以看到小鼠的频谱与患者的频谱相似,同样也涵盖了\(\gamma\)波段;
但相较而言小鼠的总体振荡比患者的稍偏低频一些。
而小鼠的能量曲线也同样在光栅开始后出现了明显的增强,并横跨了整个刺激期间。

\begin{figure}[h]
    \centering
    \includegraphics[width=\textwidth]{src/figures/ephys_examples.pdf}
    \caption{\textbf{电生理记录通道示例}\\
    (\textbf{a})~电极的脑区定位。每个红点均为一个记录通道,这里将记录到的
    6位患者的电极画在一起。(\textbf{b}, \textbf{c}, \textbf{d})~不同行为范式下
    同一个通道的示例。b: 脑电信号的原始形式,黑色的竖线对应光栅定时的时间。
    c: 脑电信号通过小波变化得到的能量频谱,展现了信号能量主要集中于\(\gamma\)波段。
    d: 带通滤波后的脑电信号能量曲线。
    (\textbf{e}, \textbf{f}, \textbf{g})~小鼠中局部场电位的示例。}
    \label{fig:ephys_example}
\end{figure}

\subsection{能量曲线的变化趋势}
依据患者的术前MRI和术后CT,我们得以对每个电极进行定位。
我们从患者\#1中选择了通道数目最多的三个脑区,
视觉皮层(visual cortex, 主要为Brodmann分区17、18、19, 共计32个通道),
扣带回(cingulate cortex, 主要为Brodmann分区30、31,共计18个通道),
以及颞中回(middle temporal gyrus, 主要为Brodmann分区21,共计10个通道)。

对相同脑区下的脑电能量曲线进行平均后(图\ref{fig:ephys_network}~a),
我们发现三个脑区的能量曲线形态十分相似,提示光栅视觉刺激可能会引起全脑范围的活动。
另一方面,对于三种不同的行为范式,轻拍手和默想下的能量曲线基本一直,而空想下的
能量曲线较前两者有明显的下降。%TODO statistical significance
提示注意力的集中程度可能会影响大脑的整体活动,但需要额外的实验进行验证和探究。

%扩散方向

\subsection{相位分析显示学习过程}
脑电信号可以被看作是一种特殊的数字信号,因而我们进一步计算了脑电信号的相位信息。
不同于信号的能量,相位的绝对值没有实际的生物学意义,但相位可以被认为
是一种神经元群体放电的状态\cite{gu2010phase}。
由于相位的变化通常体现在较低的频段,我们选取了\(\delta\)波段(\(0.5 \sim 4\ \text{Hz}\))作为相位分析的对象。
为了量化脑电信号的相位,我们将同一天中20次刺激下的脑电信号分为一组,
计算了各组的ITPC值,并对各个通道的ITPC值依照脑区进行平均(图\ref{fig:ephys_network}~b)。
可见,患者的脑电出现了两次相位锁定(phase locking),分别位于光栅开始和光栅结束。
其中,对于轻拍手和模型范式,ITPC值基本一致;
而空想下,ITPC值较前两者明显降低。%TODO statistical significance
提示在空想下的放电活动的一致性不如轻拍手和默想范式。

同时,我们也对小鼠的脑电信号做了相同的相位分析(图\ref{fig:ephys_network}~d)。
与患者的相位不同,小鼠的脑电只有在光栅开始后有一次相位锁定,而在光栅结束后没有。

此外,我们将轻拍手范式下,依据患者是否存在提前打手(即,是否能够较为准确的预判时间间隔),
将脑电信号分为两组,即提前打手组和延后打手组。我们对这两组的脑电信号分别计算了
各自的ITPC值(图\ref{fig:ephys_network}~c),发现延后打手的行为中,脑电信号的
相位没有出现提前打手下的相位锁定现象。提示能否准确预测时间间隔,
可能回造成相同视觉刺激下的不同放电模式。

\begin{figure}[h]
    \centering
    \includegraphics[width=\textwidth]{src/figures/ephys_network.pdf}
    \caption{\textbf{不同脑区下能量曲线与相位锁定}\\
    (\textbf{a})~不同脑区不同行为范式下的能量曲线。
    (\textbf{b})~不同脑区不同行为范式下的相位变化。
    (\textbf{c})~轻拍手范式下,提前打手(准确预判)和延后打手(错误预判)的相位变化。
    (\textbf{d})~小鼠场电位的相位变化。}
    \label{fig:ephys_network}
\end{figure}



\section{理论机制探究}

小鼠和人的行为学实验结果都显示了,对于规律性时间间隔的刺激,
被试需要对其进行一定的学习和适应过程。尤其是对于人的实验结果,
我们可以看到被试通常在前四个刺激(第一个五份)内就已经能够较为
准确的预测我们设定的时间间隔了。我们可以将这一学习过程看作是
强化学习(reinforcement learning)的过程: 每次光栅刺激小鼠都会
得到相应的水作为反馈,而人则可以直接将是否正确完成任务作为反馈
来调整自己的行为。

学习的过程通常与突触间的可塑性相关联,而人的学习过程十分短暂,
提示可能与短时程突触可塑性(STP)有关;而小鼠的行为在七天训练中
逐渐好转,提示其学习过程可能与长时程突触可变性(LTP)相关。
而在感觉皮层中,最重要的一类兴奋性神经元是锥体神经元(Pyramidal neuron),
其最大的特点在于树突可以分为两大部分,即尖端树突(apical dendrites)和
基树突(basal dendrites)。其尖端树突通常接受来自其他脑区的投射,
而基树突主要与周围神经元进行交流,而突触棘在树突上的空间分布通常决定着
神经元的选择性\cite{mel2017synaptic}。
突触后电位在树突上的整合通常是非线性的,
也有实验研究表明尖端树突对胞体的放电主要其调制(modulation)的作用,
而基树突接受的信号才是决定胞体放电特异性的主要因素\cite{caze2017dendrites,branco2011synaptic}。
尖端树突接受的来源往往十分广泛而复杂,有实验表明其来源除了与其相关
的感觉输入外,还与刺激相关的时间信息,对刺激的预测,刺激后的反馈(奖赏)
等多种因素相关\cite{lacefield2019reinforcement}。

脑电信号的能量通常可以与该区域内神经元的发电强弱相对应。
而患者在空想范式下脑电信号能量也有上升,但幅度较其他两种范式小(图\ref{fig:ephys_network}~a)。
这可能是由于来自其他脑区的信号在注意力减弱是也相应减弱,通过
尖端突触的调制作用而产生的。但具体的机制仍需要更详尽的实验加以验证。

然而,脑电信号的相位很难和生物学上的意义相对应,而只能抽象的认为是该区域内
神经元放电的某种模式或状态。在患者轻拍手的范式下,提前打手和延后打手的
脑电信号相位有着明显的不同(图\ref{fig:ephys_network}~c)。
这可能是由于只有视觉刺激时,神经元放电不是很协调;而当与预测相关的信号
同时到达尖端突触时,神经元的放电会被协调而呈现一定的模式(图\ref{fig:theory})。

\begin{figure}[h]
    \centering
    \includegraphics[width=\textwidth]{src/figures/theory.pdf}
    \caption{\textbf{可能的理论机制}\\
    在最初还没有学会时间间隔时,被试对刺激没有预测,如第二列所示。
    学会时间间隔后,若正确做出预判,神经元的状态可能会较没有预测时不同,如第三列所示。
    倘若预测出现失误,则神经元的状态可能而学习前类似,如第四列所示。}
    \label{fig:theory}
\end{figure}




% 行为学结果
%% weber定律
%% 增强学习的过程
%% 与小鼠的比对

% 脑电信号
%% 能量谱分析
%% 带通滤波能量谱与行为的关联
%% 相位分析
%% 与小鼠的对比

% 行为与脑电的关联
%% 支持向量机
%% 回溯神经网络模拟
%% 对神经系统网络的窥探

\chapter{讨论与展望}

在本课题中,我们依托与华山医院的合作,让患者做了与小鼠实验相对应的心理学实验,
并同时采集了来自患者的脑电信号。通过对行为学的分析,以及脑电信号的能量分解和
相位分析,我们得到了一些初步的结论。根据行为学的表现,我们可以初步判断小鼠和
人对于规律性视觉刺激的都有提前预测的能力,但都需要通过一定的学习和适应。
但小鼠的学习过程相较而言更长,需要数天的时间行为才会有明显的提升;而人在学习
我们的范式时通常只需要几次刺激即可。人的快速学习时间间隔的能力,也同样在其他的
研究中体现,与前人的结果相一致\cite{simen2011model}。对于脑电的信号而言,
我们从\(\gamma\)波段的能量上来看,主要是由视觉刺激主导和诱发的。而能量的变化上,
人和小鼠的结果基本一致。另一方面,我们通过对相位的分析,可以看到人在正确预测
时间间隔和错误预测之间的相位存在着较为显著的差别,提示了其变化可能与行为的结果
和反馈相关;而相位也间接的体现了脑电时间预测信号对感觉皮层的影响。

其次,本课题也通过相同的视觉刺激和相近的行为学范式,让人和小鼠的时间结果具有
一定的可比性。对于人体本身的机制和原理,一直是神经科学和其他生物科学,尤其是
基础医学想解答的问题所在。而由于伦理等因素,我们无法直接对正常人体做各种侵入性和
创伤性的实验;我们也因此诞生了而许多的模式生物。
在这里,我们就可以对模式生物中的一种,小鼠,和人进行对比。我们可以看到在脑电局部场电位
的表现上,小鼠的波段较人的波段更接近与低频。在相位上,小鼠仅有一次对光起始反应的
相位锁定,而人却有对光起始和终止反应的两次相位锁定。两一方面,人在行为学上的表现
也提示了人脑的学习能力比小鼠高很多。我们的结果提示了人脑的工作机理和小鼠的工作原理
可能存在着一些较大的区别。
而前人的研究也发现在灵长类动物的初级视皮层中,存在着功能柱的结构\cite{};
在小鼠的初级视皮层里却没有明显的功能柱,而呈现弥散分布的结构\cite{}。

本课题围绕规律性刺激下的时间预测信号在大脑中的体现和影响,对植入有电极的患者
进行了一系列的心理学实验。我们也将人的脑电结果与小鼠相近范式下的电生理结果做了
一定的对比。然而SEEG所能获得的脑电信号还是过于广泛,而信噪比也无法与实验室条件下
的电生理记录相类比;因而现象背后更精细的机制,尤其是细胞层面的网络机制还需要
额外的实验来加以验证。

% 行为本身带来的科学问题
% 人类与模式生物之间的对比和展望

% ---------------- Appendix ----------------
%\appendix

%\renewcommand{\thechapter}{附录{\Alph{chapter}}}

% ----------------
\chapter{综述}

\begin{center}
\textbf{\sanhao{计时(timing)相关理论模型}}
\end{center}

\bigskip
\noindent \textbf{摘要: \hspace{\Han}}
计时是神经系统一个重要的功能,不同的精度的计时有着不同的生物学机制;
其中毫秒到秒层次的计时原理还不甚明朗。本综述回顾了三种常见的计时理论模型,
包括振荡模型,坡度模型和动态系统。回顾了不同模型的主要特点和解决的问题,以及
各自主要的不足之处。最后强调了振荡模型和坡度模型可能只是动态系统的读出结果,
而动态系统对实验结果的解释最为多态和全面,但其网络构建和训练算法与神经系统
实际的情况可能仍有所出入。

\bigskip
\noindent \textbf{关键词: \hspace{\Han}}
计时;\;
神经震荡;\;
坡度模型;\;
动态系统;\;


% 引言
\section{引言}

% Timing and everyday life, why it is so important
%时间就好像空间一样,也是现实世界的重要组成部分。
%因而对于时间有好的掌控和理解,对于每个个体的学习、记忆、行为都至关重要。
时间和空间一样,也是现实世界的重要组成部分。而时间又同空间有本质上的差别,
时间不能像空间一样朝任意方向随意的移动,而只能朝一个方向以固定的速率前进。
因而生物体对时间感知无法像对空间一样具有主动的探索性或者多样性,
例如视觉中的方向选择性,海马里的place cell等。
另一方面,时间作为现实的一个维度,对视觉、听觉、触觉等等几乎所有的感觉和运动
甚至高级的皮层功能都是至关重要的,因而生物体神经系统对时间信息的编码应当
是普遍而广泛的。

% timing has different scales, it is the millisecond and second timing we are
% talking about.
不同于我们人造的时钟,生物体内对不同尺度的计时有着截然不同的生物学机制。
例如,微秒层次的计时,依赖于不同树突棘接受的动作电位到达树突的微小时间差异来实现;
生物体也主要用此来定位声音的来源\cite{moiseff1981neuronal}。
例如,以天计的生物钟,独立于动作电位依靠
转录、合成、降解的动态循环来实现;掌控着生物体的昼夜节律\cite{panda2002circadian}。
神经生物发展至今,已经发现和明确了许多机制,但对于毫秒和数秒层面计时的机制依然不是很明朗
\cite{buonomano2007biology,paton2018neural}。
而这个层次的计时也最为复杂和重要。它可以帮助生物体对即将到来的事件作出预判,
协助不同个体间的交流,让我们有能力创造出音乐,有能力用语言来交流等等。

% sensory timing and motor timing are different, but more and more articles
% talk about sensorimotor coupling and showing that they might come from the
% same neural circuit.
%对于毫秒和数秒层面计时,在过去常常分为感觉计时(sensory timing)和
%运动计时(motor timing)。感觉计时更关注于神经网络如何对外部刺激侦测
%并提取出时间序列,而运动时间更关注于如何主动的产生时间序列和预判的信号。
%随着研究的不断推进,人们发现在很多脑区和核团存在感觉运动的耦连(sensorimotor coupling)
对于时间感知和计时的理论模型大致可以分为两大类,即内在模型(intrisinc model)和
专用模型(dedicated model)\cite{ivry2008dedicated,paton2018neural}。
专用模型的主要观点在于大脑中存在特定的管理计时的区域,就好像视觉皮层主管视觉,
听觉皮层主管听觉一样,计时也存在特殊的计时相关皮层或核团。
而内在模型的主要观点在于计时是一切行为和功能所必需的一环,因而计时是广泛而普遍的,
不存在特定的计时脑区,而是分散于不同的皮层和核团,与不同的脑区所主要负责的功能相整合在一起。
%而随着各类研究的进行,有越来越多的证据表明内在模型可能更加符合实际情况。

% Timing has many unique properties, like weber's law and time warp.
% skip this part.

% for this review, we will talk about sensory timing, especially in V1.
% we will first review theoretical mechanisms for timing and some other
% empirical evidence. Finally, we will talk about some limitations of
% the current models and theories.
对于本综述,我们将主要回顾一下当前主流的对毫秒和数秒层面计时的理论机制和相关模型假说,
包括震荡模型(oscillation),缓坡模型(ramping model)以及动态系统(dynamic system);
其中,我们将着重探讨动态系统理论模型的细节。
最后,我们会进一步探讨一下当下这些机制和模型的主要优势和不足,以及未来的展望。


% 理论机制与实验依据
%\section{理论机制与相关实验研究}
对计时机制的探究可以帮助我们更好的理解学习和记忆的原理,同时对于神经系统工作的
更加通用普适的理论模型的提出也至关重要。在这里,我们主要从细胞层面和环路层面两个角度来
回顾主要的理论模型和相关实验设计。

\subsection{细胞层面}
有越来越多的实验表明存在有神经元对特定的时间间隔或者频率有特异性。而这些特异性
产生的主要因素包括了各类受体,离子通道,以及short-term synaptic plasticity。

% 离子通道

% 突出可塑性

\subsection{环路层面}
神经元与其他细胞最大的区别在于其可兴奋性,由于其特殊的膜蛋白和离子通道,让神经元
能够利用膜电位的变化来传递信息。而在我们人类的大脑里有着%TODO
神经元,而它们形成的突触的数量则更加庞大。如此庞大的通信网络彼此密不可分,
又执行着%TODO

借助于多通道电极记录的技术,让我们可以对这个混沌系统有更近距离的观察。



$$ \frac{dX}{dt} = f(X) + U $$

%% this is the most interesting thing here.

\section{震荡模型 Oscillation Model}
% what is oscillation
每个神经在发放动作电位时,都会引起其附近区域的电场发生一定的变化;
而一小块区域内的一群神经元的共同放电可以引起局部电场的波动。
这些局部电场的波动被附近的电极记录,就得到了局部场电位(Local Field Potential)的信号。
常用的记录局部场电位的方法有普通的脑电图(Electroencephalogram, EEG),
立体脑电图(Stereoelectroencephalography, SEEG), 皮层脑电图(Electrocorticography, ECoG)等。

% correlation between oscillation and timing
脑电信号也是模拟信号的一种类型,因而也存在震荡(Oscillation)。
在过去由于实验手段的限制,研究工作主要集中在局部场电位的分析;
但随着研究手段的进步,有越来越多的的证据表明与震荡相关的计时现象只是
更精细的单细胞层面的活动的总体体现\cite{paton2018neural}。
但这并不意味着神经震荡与计时毫无关联,相反神经震荡对计时有着比较强的相关性,
尤其是在与运动相关的计时中,例如呼吸节律,行走时的节律等等\cite{paton2018neural}。

大多数的震荡模型均同震荡活动的相位(Phase)相关,例如Connectionist model,
Beat frequency model等\cite{matell2004cortico}。来自不同脑区或者不同频率波段
的震荡相位随着时间的推移而出现同步的共振,或者由于刺激的出现而引起重置(reset)
是的相位处于同一位置。

% further finding and summary
震荡模型的核心在于相位的同步性检测(Coincidence Detection),但由于局部场电位的
记录范围通常比较大,因而信噪比通常不高,同时信号所反应的是神经元的群体性活动。
而后续的研究也提示震荡模型所看到的很多现象可以用单细胞层面结果进行解释。
但另一方面,震荡模型可以让我们对受伦理等限制的人体实验的结果做一定的分析和解释。

\section{梯度模型 Ramping Model}

% what is ramping phenomenon
在动物实验中,存在着神经元的放电频率会随着时间出现线性递增或者递减,直到某一个阈值
触发一定的后续行为的现象,并被称作梯度模型或者缓坡模型(Ramping Model)
\cite{durstewitz2003self,simen2011model,paton2018neural}。

% relationship between ramping and timing
这种放电频率线性变化的现象广泛分布于动物的神经系统中\cite{durstewitz2003self},


% summary

\section{动态系统 Dynamic System}

% what is dynamic system

% reservoir computation

% temporal scaling

% synaptic plasticity

% sensorimotor coupling

% summary



% 计时与初级视皮层
%\section{计时与初级视皮层}

初级视皮层(primary visual cortex, V1)再过去被认为只负责了简单的视觉信号的传递
和简单的处理,例如方向选择性等。而随着各类研究的深入,初级视皮层被发现能够
处理很多的高级功能。这也提示了大脑并不是一个严格分区分工的器官,而应该视作
一个整体;各个部分都互相连系并发挥着各自的影响。

Shuler实验室利用大鼠来对初级视皮层中的计时现象进行研究。他们首先利用训练大鼠
通过识别不同的视觉刺激来获得奖赏。而奖赏的大小与大鼠等待的时间存在一定的函数
关系。经过学习之后的大鼠可以在看到不同的刺激后,等待不同的时间再去触发奖赏。
利用多通道电生理记录,他们发现存在有神经元与计时显著相关。且神经元的活动
随着训练的增加而逐渐明显。

之后他们又利用药物干预的方法,证明了初级视皮层中的这些神经元与计时行为的好坏
存在因果关系。且学习新的计时涉及到了初级视皮层中的乙酰胆碱能相关的神经元或者突触。
随后他们又利用离体活体脑片膜片钳技术发现这些胆碱能主要来自L5/6的投射。最后他们
借助与光遗传的手段控制了从大鼠前额叶投射到初级视皮层的突触,证明了来自大鼠前额叶
的乙酰胆碱能神经元投射与初级视皮层形成新的计时行为存在因果关系。另一方面,
他们也表明大鼠的初级视皮层的活动与自主的计时行为存在相关性。


% 目前的局限所在与未来的发展
\section{小结与展望}

虽然大脑是我们人体中最为复杂和精细的器官,但它也是通过发育一点点成长而来;
从经济的角度而言,不同的感官不同的刺激不同的行为需要完全不同的算法和实现方法是
不经济的。而我们的大脑有着十分庞大的冗余量,不同的功能区也可以出现不同程度的相互替代,
这些都提示了大脑工作的背后可能存在着一个通用的算法,来支配着整个大脑的学习和活动。
另一方面,Saining Xie等人对神经网络的研究发现完全随机的神经网络有着和人为设计的
网络相似的正确率,有时甚至会比人为设计的网络有着更高的效率\cite{xie2019exploring}。
虽然这是一个纯理论的计算科学的研究,但也从它的角度为我们的神经科学提供了一些可能。

本综述简单的回顾了计时领域中常见的三大类理论模型: (1)振荡模型; (2)缓坡模型; (3)动态系统。
每种模型都是对各自获得的实验结果的拟合。从过去的EEG和局部场电位获得的振荡模型,
到之后基于单通道电信号的缓坡模型,和多通道电生理记录的动态系统。每种模型都有各自的优势
和可以用来解释的现象。而振荡模型和坡度模型所描述的现象可能只是动态系统的读出结果,
同时动态系统所描述的现象不仅可以用在计时中,也可以推广向不同的感觉和运动信息的处理。
因而相较而言,动态系统所描述的原理可能更接近于神经系统处理信息的方式方法。
但我们依然需要更多的实验和理论探究来解决大脑动态网络的出现原理和学习过程中大脑是如何
实现快速动态的调整等问题。




% ---------------- Back Matter ----------------
\backmatter
\setlength{\baselineskip}{10pt}
\phantomsection
\addcontentsline{toc}{chapter}{\bibname}
%\bibliographystyle{FDUbib}
\bibliography{ThesisBib}

\backchapter{致谢}

\clearpage
%\printindex

\end{document}
