\chapter{前言}

中枢神经系统对时间的编码和感知与生物的行为和感觉密切相关。
而对每一个个体的生存而言,预判未来可能发生的事件更是至关重要。
过去的研究主要集中于模式动物对时间的感知与预测上;由于技术手段和伦理等的限制,对人脑的研究相对较少。
另一方面,过去的研究大多将感觉与运动的计时信号分别对待研究;
而最近几年的科学研究对感觉和运动的信息整合愈加重视。

对于中枢神经系统对时间的感知,大致可分为两个大的派系:专用模型(dedicated model)和内在模型(intrinsic model);
而越来越的实验证据与内在模型的描述相符合\cite{paton_neural_2018}。
对于内在模型,最显著的一个特点就是时间的感知和编码是由神经元群体(population)来完成,
而群体内每一个单元的动态状态在时间尺度上共同组成了神经活动轨迹(neural trajectory)
\cite{buonomano_state-dependent_2009, remington_dynamical_2018}。

最初,人们认为初级视皮层只参与了视觉感觉信息处理的最初阶段。
但随着研究的深入,有越来越多的证据表明初级视皮层可能也参与了其他高级的脑功能。
Salvioni等人通过经颅磁刺激(TMS)技术探究了正常人初级视皮层和纹外皮层对计时相关活动的作用;
发现抑制了视皮层后,人在进行与计时相关的视觉任务时,正确率会显著下降;而与计时无关的视觉任务则不受影响\cite{salvioni_how_2013}。
在非人灵长类动物的实验中,Lima 等人发现初级视皮层的局部场电位(LFP)会与动物计时行为出现相关的变化,
包括alpha波段振荡的减弱和gamma波段振荡的增强\cite{lima_gamma_2011}。
而在啮齿类模式生物中,Shuler等人发现在初级视皮层中存在与奖励时间间隔相关的神经活动以及场电位变化\cite{chubykin_cholinergic_2013,shuler_reward_2006,zold_theta_2015};
之后,他们又通过药物阻断、膜片钳、光遗传等手段进一步验证了乙酰胆碱能神经元输入对初级视皮层的增强学习过程有决定性的作用\cite{chubykin_cholinergic_2013,liu_selective_2015,namboodiri_visually_2015}。

此课题我们将研究的重点关注在相同视觉刺激下人脑和小鼠脑场电位(local fiedl potential, LFP)的比较,以及人脑场电位在时间预测性行为下的变化。
本课题将依托导师与华山医院的合作,收集sEEG术后癫痫患者的脑电数据,并平行地与过去相同范式刺激下的小鼠电生理数据做比较。
