\chapter{讨论与展望}

在本课题中,我们依托与华山医院的合作,让患者做了与小鼠实验相对应的心理学实验,
并同时采集了来自患者的脑电信号。通过对行为学的分析,以及脑电信号的能量分解和
相位分析,我们得到了一些初步的结论。根据行为学的表现,我们可以初步判断小鼠和
人对于规律性视觉刺激的都有提前预测的能力,但都需要通过一定的学习和适应。
但小鼠的学习过程相较而言更长,需要数天的时间行为才会有明显的提升;而人在学习
我们的范式时通常只需要几次刺激即可。人的快速学习时间间隔的能力,也同样在其他的
研究中体现,与前人的结果相一致\cite{simen2011model}。对于脑电的信号而言,
我们从\(\gamma\)波段的能量上来看,主要是由视觉刺激主导和诱发的。而能量的变化上,
人和小鼠的结果基本一致。另一方面,我们通过对相位的分析,可以看到人在正确预测
时间间隔和错误预测之间的相位存在着较为显著的差别,提示了其变化可能与行为的结果
和反馈相关;而相位也间接的体现了脑电时间预测信号对感觉皮层的影响。

其次,本课题也通过相同的视觉刺激和相近的行为学范式,让人和小鼠的时间结果具有
一定的可比性。对于人体本身的机制和原理,一直是神经科学和其他生物科学,尤其是
基础医学想解答的问题所在。而由于伦理等因素,我们无法直接对正常人体做各种侵入性和
创伤性的实验;我们也因此诞生了而许多的模式生物。
在这里,我们就可以对模式生物中的一种,小鼠,和人进行对比。我们可以看到在脑电局部场电位
的表现上,小鼠的波段较人的波段更接近与低频。在相位上,小鼠仅有一次对光起始反应的
相位锁定,而人却有对光起始和终止反应的两次相位锁定。两一方面,人在行为学上的表现
也提示了人脑的学习能力比小鼠高很多。我们的结果提示了人脑的工作机理和小鼠的工作原理
可能存在着一些较大的区别。
而前人的研究也发现在灵长类动物的初级视皮层中,存在着功能柱的结构\cite{};
在小鼠的初级视皮层里却没有明显的功能柱,而呈现弥散分布的结构\cite{}。

本课题围绕规律性刺激下的时间预测信号在大脑中的体现和影响,对植入有电极的患者
进行了一系列的心理学实验。我们也将人的脑电结果与小鼠相近范式下的电生理结果做了
一定的对比。然而SEEG所能获得的脑电信号还是过于广泛,而信噪比也无法与实验室条件下
的电生理记录相类比;因而现象背后更精细的机制,尤其是细胞层面的网络机制还需要
额外的实验来加以验证。
