\chapter{综述}

\begin{center}
\textbf{\sanhao{计时(timing)相关理论模型}}
\end{center}

\bigskip
\noindent \textbf{摘要: \hspace{\Han}}
计时是神经系统一个重要的功能,不同的精度的计时有着不同的生物学机制;
其中毫秒到秒层次的计时原理还不甚明朗。本综述回顾了三种常见的计时理论模型,
包括振荡模型,缓坡模型和动态系统。回顾了不同模型的主要特点和解决的问题,以及
各自主要的不足之处。最后强调了振荡模型和坡度模型可能只是动态系统的读出结果,
而动态系统对实验结果的解释最为多态和全面,但其网络构建和训练算法与神经系统
实际的情况可能仍有所出入。

\bigskip
\noindent \textbf{关键词: \hspace{\Han}}
计时;\;
神经震荡;\;
缓坡模型;\;
动态系统;\;


% 引言
\section*{引言}
\addcontentsline{toc}{section}{引言}
\section{引言}

% Timing and everyday life, why it is so important
%时间就好像空间一样,也是现实世界的重要组成部分。
%因而对于时间有好的掌控和理解,对于每个个体的学习、记忆、行为都至关重要。
时间和空间一样,也是现实世界的重要组成部分。而时间又同空间有本质上的差别,
时间不能像空间一样朝任意方向随意的移动,而只能朝一个方向以固定的速率前进。
因而生物体对时间感知无法像对空间一样具有主动的探索性或者多样性,
例如视觉中的方向选择性,海马里的place cell等。
另一方面,时间作为现实的一个维度,对视觉、听觉、触觉等等几乎所有的感觉和运动
甚至高级的皮层功能都是至关重要的,因而生物体神经系统对时间信息的编码应当
是普遍而广泛的。

% timing has different scales, it is the millisecond and second timing we are
% talking about.
不同于我们人造的时钟,生物体内对不同尺度的计时有着截然不同的生物学机制。
例如,微秒层次的计时,依赖于不同树突棘接受的动作电位到达树突的微小时间差异来实现;
生物体也主要用此来定位声音的来源\cite{moiseff1981neuronal}。
例如,以天计的生物钟,独立于动作电位依靠
转录、合成、降解的动态循环来实现;掌控着生物体的昼夜节律\cite{panda2002circadian}。
神经生物发展至今,已经发现和明确了许多机制,但对于毫秒和数秒层面计时的机制依然不是很明朗
\cite{buonomano2007biology,paton2018neural}。
而这个层次的计时也最为复杂和重要。它可以帮助生物体对即将到来的事件作出预判,
协助不同个体间的交流,让我们有能力创造出音乐,有能力用语言来交流等等。

% sensory timing and motor timing are different, but more and more articles
% talk about sensorimotor coupling and showing that they might come from the
% same neural circuit.
%对于毫秒和数秒层面计时,在过去常常分为感觉计时(sensory timing)和
%运动计时(motor timing)。感觉计时更关注于神经网络如何对外部刺激侦测
%并提取出时间序列,而运动时间更关注于如何主动的产生时间序列和预判的信号。
%随着研究的不断推进,人们发现在很多脑区和核团存在感觉运动的耦连(sensorimotor coupling)
对于时间感知和计时的理论模型大致可以分为两大类,即内在模型(intrisinc model)和
专用模型(dedicated model)\cite{ivry2008dedicated,paton2018neural}。
专用模型的主要观点在于大脑中存在特定的管理计时的区域,就好像视觉皮层主管视觉,
听觉皮层主管听觉一样,计时也存在特殊的计时相关皮层或核团。
而内在模型的主要观点在于计时是一切行为和功能所必需的一环,因而计时是广泛而普遍的,
不存在特定的计时脑区,而是分散于不同的皮层和核团,与不同的脑区所主要负责的功能相整合在一起。
%而随着各类研究的进行,有越来越多的证据表明内在模型可能更加符合实际情况。

% Timing has many unique properties, like weber's law and time warp.
% skip this part.

% for this review, we will talk about sensory timing, especially in V1.
% we will first review theoretical mechanisms for timing and some other
% empirical evidence. Finally, we will talk about some limitations of
% the current models and theories.
对于本综述,我们将主要回顾一下当前主流的对毫秒和数秒层面计时的理论机制和相关模型假说,
包括震荡模型(oscillation),缓坡模型(ramping model)以及动态系统(dynamic system);
其中,我们将着重探讨动态系统理论模型的细节。
最后,我们会进一步探讨一下当下这些机制和模型的主要优势和不足,以及未来的展望。


% 理论机制与实验依据
%\section{理论机制与相关实验研究}
对计时机制的探究可以帮助我们更好的理解学习和记忆的原理,同时对于神经系统工作的
更加通用普适的理论模型的提出也至关重要。在这里,我们主要从细胞层面和环路层面两个角度来
回顾主要的理论模型和相关实验设计。

\subsection{细胞层面}
有越来越多的实验表明存在有神经元对特定的时间间隔或者频率有特异性。而这些特异性
产生的主要因素包括了各类受体,离子通道,以及short-term synaptic plasticity。

% 离子通道

% 突出可塑性

\subsection{环路层面}
神经元与其他细胞最大的区别在于其可兴奋性,由于其特殊的膜蛋白和离子通道,让神经元
能够利用膜电位的变化来传递信息。而在我们人类的大脑里有着%TODO
神经元,而它们形成的突触的数量则更加庞大。如此庞大的通信网络彼此密不可分,
又执行着%TODO

借助于多通道电极记录的技术,让我们可以对这个混沌系统有更近距离的观察。



$$ \frac{dX}{dt} = f(X) + U $$

%% this is the most interesting thing here.

\section*{振荡模型}
\addcontentsline{toc}{section}{振荡模型}
% what is oscillation
每个神经在发放动作电位时,都会引起其附近区域的电场发生一定的变化;
而一小块区域内的一群神经元的共同放电可以引起局部电场的波动。
这些局部电场的波动被附近的电极记录,就得到了局部场电位(Local Field Potential)的信号。
常用的记录局部场电位的方法有普通的脑电图(Electroencephalogram, EEG),
立体脑电图(Stereoelectroencephalography, SEEG), 皮层脑电图(Electrocorticography, ECoG)等。

% correlation between oscillation and timing
脑电信号也是模拟信号的一种类型,因而也存在震荡(Oscillation)。
在过去由于实验手段的限制,研究工作主要集中在局部场电位的分析;
但随着研究手段的进步,有越来越多的的证据表明与震荡相关的计时现象只是
更精细的单细胞层面的活动的总体体现\citereview{paton2018neural}。
但这并不意味着神经震荡与计时毫无关联,相反神经震荡对计时有着比较强的相关性,
尤其是在与运动相关的计时中,例如呼吸节律,行走时的节律等等\citereview{paton2018neural}。

大多数的震荡模型均同震荡活动的相位(Phase)相关,例如Connectionist model,
Beat frequency model等\citereview{matell2004cortico}。来自不同脑区或者不同频率波段
的震荡相位随着时间的推移而出现同步的共振,或者由于刺激的出现而引起重置(reset)
是的相位处于同一位置。

% further finding and summary
震荡模型的核心在于相位的同步性检测(Coincidence Detection),但由于局部场电位的
记录范围通常比较大,因而信噪比通常不高,同时信号所反应的是神经元的群体性活动。
而后续的研究也提示震荡模型所看到的很多现象可以用单细胞层面结果进行解释。
但另一方面,震荡模型可以让我们对受伦理等限制的人体实验的结果做一定的分析和解释。

\section*{缓坡模型}
\addcontentsline{toc}{section}{缓坡模型}
\section{梯度模型 Ramping Model}

% what is ramping phenomenon
在动物实验中,存在着神经元的放电频率会随着时间出现线性递增或者递减,直到某一个阈值
触发一定的后续行为的现象,并被称作梯度模型或者缓坡模型(Ramping Model)
\cite{durstewitz2003self,simen2011model,paton2018neural}。

% relationship between ramping and timing
这种放电频率线性变化的现象广泛分布于动物的神经系统中\cite{durstewitz2003self,simen2011model},
Simen等人对该现象进行了数学建模,
他们将神经元的活动简化为泊松进程(Poisson process),并将模型分为了四个层次:
(1)泊松进程拟合的单个神经元活动; (2)兴奋性和抑制性强度一致的细胞群体;
(3)不同细胞群体组成的积分器(integrator); (4)将所有环路输出成结果的输出层,模拟行为的结果\cite{simen2011model}。
他们的模拟结果与实际的实验结果十分符合,同时也可以用模型解释韦伯定律。
对于不同的时长任务,相同的体现计时的神经元的放电频率变化的斜率会发生相对应的改变,
以是得最后到达阈值的时刻与实际的时长相接近。
而Durstewitz利用了spiking leaky-integrate-and-fire (LIF) neuron model,
对梯度模型的建模则更加贴近真实的神经元活动\cite{durstewitz2003self}。
虽然模型没有分多个层次,但对单细胞的膜电位模拟更加精准和贴近现实;
而得到的结果也十分类似,同样也得到了放电频率的线性变化。

% summary
虽然在实验和理论模型中都可以看到神经元放电频率的线性变化与计时行为相关,
但值得注意的是并不是所有的神经元都存在这类变化。而在理论模型中,所有的
神经元都对结果起着作用;因而梯度模型所描述的放电频率线性变化并非是说
其他没有线性变化的神经元就不参与了计时行为;而应该理解为线性变化更容易被
发现和量化。而线性变化可能和Simen等人模型中的最后输出一样,也是对更低层次
活动的读取和整合。

\section*{动态系统}
\addcontentsline{toc}{section}{动态系统}
\section{动态系统 Dynamic System}

% what is dynamic system

% reservoir computation

% temporal scaling

% synaptic plasticity

% sensorimotor coupling

% summary



% 计时与初级视皮层
%\section{计时与初级视皮层}

初级视皮层(primary visual cortex, V1)再过去被认为只负责了简单的视觉信号的传递
和简单的处理,例如方向选择性等。而随着各类研究的深入,初级视皮层被发现能够
处理很多的高级功能。这也提示了大脑并不是一个严格分区分工的器官,而应该视作
一个整体;各个部分都互相连系并发挥着各自的影响。

Shuler实验室利用大鼠来对初级视皮层中的计时现象进行研究。他们首先利用训练大鼠
通过识别不同的视觉刺激来获得奖赏。而奖赏的大小与大鼠等待的时间存在一定的函数
关系。经过学习之后的大鼠可以在看到不同的刺激后,等待不同的时间再去触发奖赏。
利用多通道电生理记录,他们发现存在有神经元与计时显著相关。且神经元的活动
随着训练的增加而逐渐明显。

之后他们又利用药物干预的方法,证明了初级视皮层中的这些神经元与计时行为的好坏
存在因果关系。且学习新的计时涉及到了初级视皮层中的乙酰胆碱能相关的神经元或者突触。
随后他们又利用离体活体脑片膜片钳技术发现这些胆碱能主要来自L5/6的投射。最后他们
借助与光遗传的手段控制了从大鼠前额叶投射到初级视皮层的突触,证明了来自大鼠前额叶
的乙酰胆碱能神经元投射与初级视皮层形成新的计时行为存在因果关系。另一方面,
他们也表明大鼠的初级视皮层的活动与自主的计时行为存在相关性。


% 目前的局限所在与未来的发展
\section*{小结与展望}
\addcontentsline{toc}{section}{小结与展望}
\section{小结与展望}

虽然大脑是我们人体中最为复杂和精细的器官,但它也是通过发育一点点成长而来;
从经济的角度而言,不同的感官不同的刺激不同的行为需要完全不同的算法和实现方法是
不经济的。而我们的大脑有着十分庞大的冗余量,不同的功能区也可以出现不同程度的相互替代,
这些都提示了大脑工作的背后可能存在着一个通用的算法,来支配着整个大脑的学习和活动。
另一方面,Saining Xie等人对神经网络的研究发现完全随机的神经网络有着和人为设计的
网络相似的正确率,有时甚至会比人为设计的网络有着更高的效率\cite{xie2019exploring}。
虽然这是一个纯理论的计算科学的研究,但也从它的角度为我们的神经科学提供了一些可能。

本综述简单的回顾了计时领域中常见的三大类理论模型: (1)振荡模型; (2)梯度模型; (3)动态系统。
每中模型都是对各自获得的实验结果的拟合和模拟。从过去的脑电和局部场电位变化获得的振荡模型,
到之后基于单通道电生理记录的梯度模型,和多通道电生理记录的动态系统。每种模型都有各自的优势
和可以用来解释的现象。而振荡模型和坡度模型所描述的现象可能只是动态系统的读出结果,
同时动态系统所描述的现象不仅可以用在计时中,也可以推广向不同的感觉和运动信息的处理。
因而相较而言,动态系统所描述的原理可能更接近于神经系统处理信息的方式方法。
但我们依然需要更多的实验和理论探究来解决大脑动态网络的出现原理和学习过程中大脑是如何
实现快速动态的调整等问题。



\bibliographystylereview{gbt7714-plain}
\addcontentsline{toc}{section}{综述参考文献}
\bibliographyreview{ThesisBib}
