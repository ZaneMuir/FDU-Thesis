
% what is dynamic system
正如在缓坡模型中线性变化可能只是一群细胞活动的读出结果,
因而计时的现象可能是由于群体细胞的共同变化而实现的。
而将一群神经元看作一个动态系统,就可以来探究群体神经元活动与行为间的关联了。

% reservoir computation
神经元的群体活动可以被抽象成回溯神经网络(recurrent neural network)或者
一组微分方程(方程\ref{equ:dynamic_system})\citereview{remington2018dynamical}:
\begin{equation}
    \label{equ:dynamic_system}
    \frac{dX}{dt} = f(X) + U
\end{equation}
其中$X$代表了群体中每个神经元的状态,而$\frac{dX}{dt}$代表了神经元状态的变化,
$f(X)$体现了神经元之间的相互作用,$U$则代表了来自群体外部的刺激或者输入。

在实验过程中通常可以同时记录到数百个神经元的同时活动,而在理论模拟中也通常会选用
数百或数前个神经元作为研究对象。为了更好的来描述群体神经元的变化,
我们需要将多维度的群体做一定的降维处理(每个神经元个体看作是一个单独的维度),
最常用的方法就是主成分分析(principle compoenent analysis, PCA)。
而群体活动在低维空间中的投影常常被称作神经轨迹(neural trajectory)
\citereview{remington2018dynamical,paton2018neural}。

Hardy和Buonomano通过理论模拟,证明了同一个回溯神经网络可以编码不同的神经轨迹,
并对不同的外界刺激做出不同的反应\citereview{hardy2018encoding}。同时也有越来越多的
实验结果表明神经元的群体活动可以编码计时行为,它们的神经轨迹在相同的刺激下也总是
相吻合\citereview{paton2018neural,remington2018dynamical,bakhurin2017differential}。

神经元群体对时间信息的编码过程可以理解为是储层计算(reservoir computation)的过程
\citereview{hardy2018encoding,paton2018neural,remington2018dynamical,maass2002real}。
对于一个随机连接的群体,得到一定的刺激后,神经元之间就存在了一定的放电模式,即
一定的神经轨迹。而随着学习和反复的刺激,这条最初的神经轨迹被慢慢的固定下来,
在一定的噪音扰动下也可以保持原来的轨迹;但这条轨迹的形状与行为或者刺激没有任何
关联,而是该网络本身就存在的相互作用。学习或者反复刺激所做的只是强化了原本就
已经存在的轨迹。而运动或者行为的输出所要做的就是对这条强化后的轨迹的读取,例如
缓坡模型中放电频率的线性变化就可以理解为神经元群体活动的神经轨迹的累积。

% temporal scaling

% synaptic plasticity

% sensorimotor coupling

% summary
虽然动态系统的模型有着更多丰富的细节和可能性,但与其他模型相比,有一个最大的不足
在于它对每个神经元都做了简化和抽象,同时细胞与细胞之间的连接也是随机连接而没有
参考实际的连接模式。这些不足给模型带来了过拟合的可能性,即只要模型中神经元的个数
足够多,模拟的结果必然可以与实验结果相吻合。同时,生物对时间间隔的感知的学习可以
很快完成,尤其是一些涉及到厌恶性的刺激\citereview{simen2011model},例如电击等。
而回溯神经网络的学习需要通过较多的训练来调整权重参数,提示了当前的训练或个学习策略
与神经系统中真实发生的样子还是存在一定的差异。

