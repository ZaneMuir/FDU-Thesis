
虽然大脑是我们人体中最为复杂和精细的器官,但它也是通过发育一点点成长而来;
从经济的角度而言,不同的感官不同的刺激不同的行为需要完全不同的算法和实现方法是
不经济的。而我们的大脑有着十分庞大的冗余量,不同的功能区也可以出现不同程度的相互替代,
这些都提示了大脑工作的背后可能存在着一个通用的算法,来支配着整个大脑的学习和活动。
另一方面,Saining Xie等人对神经网络的研究发现完全随机的神经网络有着和人为设计的
网络相似的正确率,有时甚至会比人为设计的网络有着更高的效率\citereview{xie2019exploring}。
虽然这是一个纯理论的计算科学的研究,但也从它的角度为我们的神经科学提供了一些可能。

本综述简单的回顾了计时领域中常见的三大类理论模型: (1)振荡模型; (2)缓坡模型; (3)动态系统。
每种模型都是对各自获得的实验结果的拟合。从过去的EEG和局部场电位获得的振荡模型,
到之后基于单通道电信号的缓坡模型,和多通道电生理记录的动态系统。每种模型都有各自的优势
和可以用来解释的现象。而振荡模型和坡度模型所描述的现象可能只是动态系统的读出结果,
同时动态系统所描述的现象不仅可以用在计时中,也可以推广向不同的感觉和运动信息的处理。
因而相较而言,动态系统所描述的原理可能更接近于神经系统处理信息的方式方法。
但我们依然需要更多的实验和理论探究来解决大脑动态网络的出现原理和学习过程中大脑是如何
实现快速动态的调整等问题。

