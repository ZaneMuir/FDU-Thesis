\section{理论机制与相关实验研究}
对计时机制的探究可以帮助我们更好的理解学习和记忆的原理,同时对于神经系统工作的
更加通用普适的理论模型的提出也至关重要。在这里,我们主要从细胞层面和环路层面两个角度来
回顾主要的理论模型和相关实验设计。

\subsection{细胞层面}
有越来越多的实验表明存在有神经元对特定的时间间隔或者频率有特异性。而这些特异性
产生的主要因素包括了各类受体,离子通道,以及short-term synaptic plasticity。

% 离子通道

% 突出可塑性

\subsection{环路层面}
神经元与其他细胞最大的区别在于其可兴奋性,由于其特殊的膜蛋白和离子通道,让神经元
能够利用膜电位的变化来传递信息。而在我们人类的大脑里有着%TODO
神经元,而它们形成的突触的数量则更加庞大。如此庞大的通信网络彼此密不可分,
又执行着%TODO

借助于多通道电极记录的技术,让我们可以对这个混沌系统有更近距离的观察。



$$ \frac{dX}{dt} = f(X) + U $$

%% this is the most interesting thing here.
