% what is oscillation
每个神经在发放动作电位时,都会引起其附近区域的电场发生一定的变化;
而一小块区域内的一群神经元的共同放电可以引起局部电场的波动。
这些局部电场的波动被附近的电极记录,就得到了局部场电位(Local Field Potential)的信号。
常用的记录局部场电位的方法有普通的脑电图(Electroencephalogram, EEG),
立体脑电图(Stereoelectroencephalography, SEEG), 皮层脑电图(Electrocorticography, ECoG)等。

% correlation between oscillation and timing
脑电信号也是模拟信号的一种类型,因而也存在震荡(Oscillation)。
在过去由于实验手段的限制,研究工作主要集中在局部场电位的分析;
但随着研究手段的进步,有越来越多的的证据表明与震荡相关的计时现象只是
更精细的单细胞层面的活动的总体体现\citereview{paton2018neural}。
但这并不意味着神经震荡与计时毫无关联,相反神经震荡对计时有着比较强的相关性,
尤其是在与运动相关的计时中,例如呼吸节律,行走时的节律等等\citereview{paton2018neural}。

大多数的震荡模型均同震荡活动的相位(Phase)相关,例如Connectionist model,
Beat frequency model等\citereview{matell2004cortico}。来自不同脑区或者不同频率波段
的震荡相位随着时间的推移而出现同步的共振,或者由于刺激的出现而引起重置(reset)
是的相位处于同一位置。

% further finding and summary
震荡模型的核心在于相位的同步性检测(Coincidence Detection),但由于局部场电位的
记录范围通常比较大,因而信噪比通常不高,同时信号所反应的是神经元的群体性活动。
而后续的研究也提示震荡模型所看到的很多现象可以用单细胞层面结果进行解释。
但另一方面,震荡模型可以让我们对受伦理等限制的人体实验的结果做一定的分析和解释。
