\section{梯度模型 Ramping Model}

% what is ramping phenomenon
在动物实验中,存在着神经元的放电频率会随着时间出现线性递增或者递减,直到某一个阈值
触发一定的后续行为的现象,并被称作梯度模型或者缓坡模型(Ramping Model)
\cite{durstewitz2003self,simen2011model,paton2018neural}。

% relationship between ramping and timing
这种放电频率线性变化的现象广泛分布于动物的神经系统中\cite{durstewitz2003self,simen2011model},
Simen等人对该现象进行了数学建模,
他们将神经元的活动简化为泊松进程(Poisson process),并将模型分为了四个层次:
(1)泊松进程拟合的单个神经元活动; (2)兴奋性和抑制性强度一致的细胞群体;
(3)不同细胞群体组成的积分器(integrator); (4)将所有环路输出成结果的输出层,模拟行为的结果\cite{simen2011model}。
他们的模拟结果与实际的实验结果十分符合,同时也可以用模型解释韦伯定律。
对于不同的时长任务,相同的体现计时的神经元的放电频率变化的斜率会发生相对应的改变,
以是得最后到达阈值的时刻与实际的时长相接近。
而Durstewitz利用了spiking leaky-integrate-and-fire (LIF) neuron model,
对梯度模型的建模则更加贴近真实的神经元活动\cite{durstewitz2003self}。
虽然模型没有分多个层次,但对单细胞的膜电位模拟更加精准和贴近现实;
而得到的结果也十分类似,同样也得到了放电频率的线性变化。

% summary
虽然在实验和理论模型中都可以看到神经元放电频率的线性变化与计时行为相关,
但值得注意的是并不是所有的神经元都存在这类变化。而在理论模型中,所有的
神经元都对结果起着作用;因而梯度模型所描述的放电频率线性变化并非是说
其他没有线性变化的神经元就不参与了计时行为;而应该理解为线性变化更容易被
发现和量化。而线性变化可能和Simen等人模型中的最后输出一样,也是对更低层次
活动的读取和整合。
