\section{计时与初级视皮层}

初级视皮层(primary visual cortex, V1)再过去被认为只负责了简单的视觉信号的传递
和简单的处理,例如方向选择性等。而随着各类研究的深入,初级视皮层被发现能够
处理很多的高级功能。这也提示了大脑并不是一个严格分区分工的器官,而应该视作
一个整体;各个部分都互相连系并发挥着各自的影响。

Shuler实验室利用大鼠来对初级视皮层中的计时现象进行研究。他们首先利用训练大鼠
通过识别不同的视觉刺激来获得奖赏。而奖赏的大小与大鼠等待的时间存在一定的函数
关系。经过学习之后的大鼠可以在看到不同的刺激后,等待不同的时间再去触发奖赏。
利用多通道电生理记录,他们发现存在有神经元与计时显著相关。且神经元的活动
随着训练的增加而逐渐明显。

之后他们又利用药物干预的方法,证明了初级视皮层中的这些神经元与计时行为的好坏
存在因果关系。且学习新的计时涉及到了初级视皮层中的乙酰胆碱能相关的神经元或者突触。
随后他们又利用离体活体脑片膜片钳技术发现这些胆碱能主要来自L5/6的投射。最后他们
借助与光遗传的手段控制了从大鼠前额叶投射到初级视皮层的突触,证明了来自大鼠前额叶
的乙酰胆碱能神经元投射与初级视皮层形成新的计时行为存在因果关系。另一方面,
他们也表明大鼠的初级视皮层的活动与自主的计时行为存在相关性。
