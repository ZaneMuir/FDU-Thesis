\section{引言}
% Timing and everyday life, why it is so important
时间就好像空间一样,也是现实世界的重要组成部分。
因而对于时间有好的掌控和理解,对于每个个体的学习、记忆、行为都至关重要。

% timing has different scales, it is the millisecond and second timing we are
% talking about.
不同于我们人造的时钟,生物体内对不同尺度的计时有着截然不同的生物学机制。
例如,微秒层次的计时,依赖于动作电位到达相同树突的微小差异来实现;
生物体也主要用此来定位声音的来源。例如,以天计的生物钟,独立于动作电位依靠
转录合成降解的动态循环来实现;掌控着生物体的昼夜节律。神经生物发展至今,
已经发现和明确了许多机制,但对于毫秒和数秒层面计时的机制依然不是很明朗。
而这个层次的计时也最为复杂和重要。它可以帮助生物体对即将到来的事件作出预判,
协助不同个体间的交流,让我们有能力创造出音乐等等。

% sensory timing and motor timing are different, but more and more articles
% talk about sensorimotor coupling and showing that they might come from the
% same neural circuit.
对于毫秒和数秒层面计时,再过去常常分为感觉计时(sensory timing)和
运动计时(motor timing)。感觉计时更关注于神经网络如何对外部刺激侦测
并提取出时间序列,而运动时间更关注于如何主动的产生时间序列和预判的信号。
随着研究的不断推进,人们发现在很多脑区和核团存在感觉运动的耦连(sensorimotor coupling)

% Timing has many unique properties, like weber's law and time warp.
% skip this part.

% for this review, we will talk about sensory timing, especially in V1.
% we will first review theoretical mechanisms for timing and some other
% empirical evidence. Finally, we will talk about some limitations of
% the current models and theories.
对于本综述,我们将首先回顾一下当前主要的对毫秒和数秒层面计时的理论机制以及
相关的实验研究,再着重回顾一下初级视皮层内的计时现象和机制探究,最后探讨一下
当下这些机制的主要不足和未来的发展前景。
