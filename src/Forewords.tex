\frontchapter{中文摘要}
时间作为一个十分重要的维度, 我们的神经系统对不同跨度时间间隔的编码有不同的机制;
其中, 最为复杂的就是对秒和数秒层面的计时。我们通过对植入有电极的癫痫患者进行
与小鼠实验相同的视觉刺激和相近的行为学测试, 获得人在进行规律性时间间隔预测行为
下的行为和脑电信号。通过对行为学的分析, 结果提示人和小鼠的预测行为需要一定的学习过程,
而人的学习可以在几个刺激内完成。通过对脑电信号的能量谱分析, 我们发现视觉信号主要
引起\(\gamma\)波段的振荡, 而相位分析提示被试能否准确预测时间间隔会引起相位锁定
的不同。最后, 我们基于现有的实验结果, 提出了可能的理论猜想:
时间预测信号在局部场电位层面, 可能通过改变皮层放电潜在的相位变化而实现;
同时, 我们也对小鼠和人的脑电信号做了简单的对比。

\bigskip
\noindent \textbf{关键词: \hspace{\Han}}
时间预测;\;
\(\gamma\)振荡;\;
sEEG;\;
立体定向脑电图;\;
相位分析;\;



\bigskip
\noindent \textbf{中图分类号: \hspace{\Han}Q427}

% ----------------
\frontchapter{英文摘要}
% Time as an important dimension of reality, is encoded by nervous system,
% just like other perceptions. For different ranges of time intervals,
% nervous system has different mechanisms. Timing of seconds and minutes
% is one of the most complicated timing, and the mechanism is still unclear.
% Here, we used patients who had epilepsy with sEEG electrodes implantation
% as our subjects. We used the identical visual stimuli and similar behavior
% paradigms used in mice experiments for the subjects. The behavioral resutls
% implied that a learning process might be involed in our paradigms of both humen and mice.
% However, human subjects could learn the time interval within only few trials.
% Then, we analyzed the power spectra of sEEG recording, and we found that
% the visual stimuli caused \(\gamma\) oscillation most. And further phase analysis
% showed difference of phase locking between correct prediction of time interval
% and false prediction. Finally, based on the current results, we purposed
% a possible hypothesis of mechanism of our results: in local field potential,
% the prediction of time interval  might be encoded in the phase of recording signal.
Time as an important dimension of reality is encoded by the nervous system,
just like other perceptions. For different ranges of time intervals,
the nervous system has different mechanisms.
The timing of seconds and minutes is one of the most complicated timing,
and the mechanism is still unclear.
Here, we used patients who had epilepsy with sEEG electrodes implantation as our subjects.
We used identical visual stimuli and similar behavior paradigms used in mice experiments for the subjects.
The behavioral results implied that a learning process might be involed in our paradigms of both humans and mice.
However, human subjects could learn the time interval within only a few trials.
Then, we analyzed the power spectra of sEEG recording,
and we found that the visual stimuli caused γ oscillation most.
And further phase analysis showed the difference of phase locking
between the correct prediction of the time interval and false prediction.
Finally, based on the current results, we purposed a possible hypothesis of the mechanism of our results, that,
in local field potential, the prediction of time interval might be encoded in the phase of a recording signal.

\bigskip
\noindent \textbf{Key Words:\hspace{\Han}}
time prediction;\;
\(\gamma\) oscillation;\;
sEEG;\;
stereoelectroencephalograhpy;\;
phase analysis

\bigskip
\noindent \textbf{CLC Number:\hspace{\Han}Q427}
