\section{实验主要方法}

\subsection{视觉刺激与行为范式}

\indent 我们利用自制的Python程序,依托PsychoPy软件(DOI: 10.3758/s13428-018-01193-y)实时生成视觉刺激。
刺激主体为移动光栅(drifting grating),每次视觉刺激持续1秒钟;
光栅为方波光栅,对比度为100\%,时间频率为2Hz,空间频率为0.05周期每度。

我们共设计了3种行为范式,分别为“轻拍手”,“默想”,和“空想”。
“轻拍手“范式要求患者主动预判视觉刺激出现的时刻,并尽可能地在视觉出现前轻按电脑的空格键以记录患者的预判时间。
“默想”范式要求患者主动预判时间刺激的时刻,但无需做出行为而只需要在脑海中默念。
“空想”范式要求患者放空思想,以尽量减少主观思考和注意力的影响。

每种范式又分别有两种时间间隔,分别为5秒和10秒;
对于5秒间隔,视觉刺激持续1秒,50\%灰背景持续4秒;
对于10秒间隔,视觉刺激持续1秒,50\%灰背景持续9秒。
每种范式和每个时间间隔刺激均为20次,刺激结束前后均为20秒的50\%灰背景。

每位患者每天只进行一次实验,每次的顺序均为“轻拍手”,“默想”,和“空想”范式,
每个范式均为先10秒间隔后5秒间隔,两次间隔间有约30到60秒的休息时间。

\subsection{脑电数据处理}

\subsubsection{电极通道定位}

每个通道在脑中的定位主要依靠术前的MRI和术后的CT成像。
我们首先利用FSL程序(PMID: 21979382)将影像配准到MNI标准空间,并手动标记每个电极的位置。
我们再利用Bioelectromagnetism 工具箱(http://eeg.sourceforge.net/bioelectromagnetism.html) 在MATLAB中将MNI座标转化为Talairach座标,
并利用Talairach Daemon软件(PMID: 10912591)获得每个通道相对应的脑区位置。
为进一步方便分析,我们将各个脑区再根据各自的生理功能大致做了合并(见正文图表)。

\subsubsection{能量谱分析}

我们使用自制的Python3程序对原始SEEG数据进行后续分析处理。
我们首先将原始数据以时间刺激开始的时刻作为原点,前后各3秒(5秒间隔)或6秒(10秒间隔)进行切割。
之后,对各个片段做以复数形式Morlet小波为基础的小波变换,变换范围为1到150Hz。
将变换后的绝对值的对数形式做Z score标准化,并将当天标准化后的值平均后作为能量谱进行做图。

\subsubsection{回现(entrain)与光反应的自动检测}

原始数据首先通过FIR带通滤波对每个病人和特定脑区特征的频段进行滤波。
滤波后的信号进一步通过Hilbert转化并进行了如上述的切割。
将变换后的绝对值的对数形式做Z score标准化,并将当天标准化后的值平均后作为其特征波段的能量值(power)。

只有在有效窗口(刺激开始的时刻前后1秒或2秒)内能量值大于1.96,
同时在非有效窗口的能量值小于1.96的通道或者试验才能被认为有光反应或出现了回现(entrain)。

\subsubsection{回现(entrain)的随机水平(chance level)}

我们将同一范式,两次间隔试验的间隔作为基线(时长约为30到60秒)。
而后,我们随机生成1000个刺激起始时刻,重复上述的自动检测。
被认定为有出现了回现的次数占总测试次数的比例即为回现的随机水平。

\subsubsection{支持向量机(support vector machine, SVM)分析}
为了进一步量化神经元群体和不同脑区间对时间间隔的编码,
我们利用径向基核函数多分类支持向量机进行评估(multiclass radial-basis function kernel support vector machine )。
我们将时间间隔分割为20份(对于5秒间隔,每份为250毫秒,对于10秒间隔则为500毫秒)。
SVM的实现依赖于LIBSVM库(DOI: 10.1145/1961189.1961199)。
多分类采用了one-against-one的方法,SVM的输出为一个20维的向量,代表了SVM对于输入数据与对应的20个时刻的相似程度。
SVM的训练与测试采用了留一交叉验证(Leave-One-Out Cross Validation)的方法,对最后的SVM结果利用Pearson相关系数来做量化。

我们严格把控和限制了支持向量机的数据集。我们仅选择了在视皮层,扣带回的相关脑区,并将其安装有无光反应做了分类。
同时,为了增加对照并评估过拟合的可能,我们还对数据采取了试验内随机化和试验间随机化。
试验内随机化将同一次试验内的不同时间做随机化;试验间随机化则将相同时间点但不同试验的数据做随机化。
随机化后的SVM结果也用Pearson相关系数来做量化。

\subsubsection{相位分析}
%TODO
