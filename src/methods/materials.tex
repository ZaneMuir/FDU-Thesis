\section{实验材料}

\subsection{实验对象}
我们总共采集了6位患者的脑电数据, 均为男性(表\ref{tab:patient_info})。
患者由合作方复旦大学华山医院提供, 每位患者均有签订书面的知情同意书,
对实验的细节与科研用途表示支持与同意。本课题也符合华山医院伦理委员会的相关规定与同意。
每位患者均为癫痫病患者, 因手术治疗准备而在其颅内植入了数根医用不锈钢电极。
在实验期间, 患者均有服用抗癫痫药物, 且无癫痫发作。

由于患者病情发展, 以及癫痫病灶引起的并发症不一, 实际有效的病人为3人,
有效的通道数目共计268个。

\begin{table}[h]
    \centering
    \caption{被试基本信息}
    \label{tab:patient_info}
    \begin{tabular}{ccccccc}
        \hline\noalign{\smallskip}
        被试 & 性别 & 年龄 & 癫痫病灶 & 记录电极通道数目 & 实验时间(天) & 能否有效完成实验\\
        \hline\noalign{\smallskip}
        \#1 & 男 & 24 & 枕叶 & 112 & 7 & +\\ % zhouchen
        \#2 & 男 & 14 & 顶叶 & 111 & 2 & -\\ % shuyunfan
        \#3 & 男 & 26 & 颞叶 &  73 & 7 & +\\ % fansulong
        \#4 & 男 & 23 & 颞叶 & 138 & 7 & -\\ % wuzhenwei
        \#5 & 男 & 28 & 颞叶 &  91 & 7 & -\\ % zhangchen
        \#6 & 男 & 24 & 颞叶 &  83 & 7 & +\\ % wangjinan

        \hline\noalign{\smallskip}

    \end{tabular}
\end{table}

行为实验的对照组邀请了同实验室的3位年龄与患者相仿的男性同学参与;
对照组成员均自愿同意参与本课题, 且仅做了行为学部分的实验。
为了排除对照组被试同学对本课题了解的先入影响,
每位实验对象对间隔为13~秒的视觉刺激进行了实验, 连续测试了3~次, 仅测试1~天。

\subsection{实验动物}
动物实验部分的数据由张嘉漪课题组博士生于庆鹏完成并提供。
本实验中使用的所有动物都符合复旦大学上海医学院动物管理机构及使用委员会的相关规定,
并参照美国国立卫生研究院的实验动物管理及使用标准。
本课题中使用的实验动物均为8-16周雄性C57/BL6J小鼠, 购买自上海斯莱克实验动物有限责任公司。
用于后续数据分析的共计两只小鼠。

\subsection{实验仪器}

\subsubsection{脑电采集系统}
患者脑电数据的采集依赖于华山医院的数据采集系统,
Neurofax EEG-1200C采集系统(NIHON KOHDEN Corporation, Japan)。
脑电原始数据采样频率选用了2~kHz, 并经过了采集系统自带的0.5Hz到600Hz的带通滤波(bandpass filter),
以及50Hz的陷波滤波(notch filter)。

\subsubsection{小鼠电生理采集系统}
本课题中小鼠部分电生理数据采集使用了Spike2记录系统(Cambridge Electronic Design Limited, UK)。
主要采集了小鼠的局部场电位信号(Local Field Potential, LFP)。
采样频率为10~kHz, 为与后续患者数据分析相匹配,
我们人为地将采集到的数据降频至2~kHz。

\subsubsection{视觉刺激系统}
视觉刺激主要依靠自制程序实时生成。
对患者的实验中, 视觉刺激主要由Macbook Pro(13 inches, 2013-late, Apple Inc.)呈现。
屏幕位于患者的正前方, 距离患者眼睛约50~cm。
小鼠实验中, 视觉刺激主要由小显示器(7英寸)呈现, 屏幕位于小鼠的一侧眼前,
小鼠瞳孔与屏幕中心点的连线与屏幕平面垂直, 距离小鼠眼睛约10~cm。
