\section{实验材料}

\subsection{实验对象}
我们总共采集了4位患者的脑电数据,均为男性,年龄范围为22.5 +/- 3.7 岁。
患者由合作方复旦大学华山医院提供,每位患者均有签订书面的知情同意书,
对实验的细节与科研用途表示支持与同意。本课题也符合华山医院伦理委员会的相关规定与同意。

每位患者均为癫痫病患者,因手术治疗准备而在其颅内植入了数根医用不锈钢电极。
每位患者依据各自病情植入电极通道数为73到138个不等,四人共计435个有效通道。
在实验期间,患者均有服用抗癫痫药物,且无癫痫发作。

\subsection{实验动物}
动物实验部分由张嘉漪课题组博士生协助完成。
本实验中使用的所有动物都符合复旦大学上海医学院动物管理机构及使用委员会的相关规定,
并参照美国国立卫生研究院的实验动物管理及使用标准。
本课题中使用的实验动物均为8-16周雄性C57/BL6J小鼠,购买自上海斯莱克实验动物有限责任公司。
用于后续数据分析的共计两只小鼠。
%TODO 细节

\subsection{实验仪器}

\subsubsection{脑电采集系统}
患者脑电数据的采集依赖于华山医院的数据采集系统,Neurofax EEG-1200采集系统(NIHON KOHDEN Corporation, Japan)。
脑电原始数据采样频率选用了2000Hz,并经过了采集系统自带的0.5Hz到600Hz的带通滤波(bandpass filter),
以及50Hz的陷波滤波(notch filter)。
有三位患者连续7天参与了实验并记录了相应的脑电,只有一位患者因病情原因只记录了两天。

\subsubsection{小鼠电生理采集系统}
本课题中小鼠部分电生理数据采集使用了Spike2记录系统(CED co. UK)。
CED 系统主要用来采集局部场电位(Local Field Potential, LFP)和小范围神经元集群放电(Multiunit spiking)。
记录电极通过前置放大器 headstage 与信号采集器连接,
将大脑内的神经信号经过前置和后置放大器的二级放大,经过数模转换被记录软件系统采集。

\subsubsection{视觉刺激系统}
视觉刺激主要依靠自制的PsychoPy程序实时生成。%TODO ref
对患者的实验中,视觉刺激主要由Macbook Pro(13 inches, 2013-late, Apple Inc.)呈现。
屏幕位于患者的正前方,距离患者眼睛约50cm。
小鼠实验中,视觉刺激主要由小显示器(规格和公司)呈现,屏幕位于小鼠的一侧眼前,距离小鼠眼睛约10cm。
